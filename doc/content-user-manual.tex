%%%%%%%%%%%%%%%%%%%%%%%%%%%%%%%%%%%
% Basic formatting and settings
%%%%%%%%%%%%%%%%%%%%%%%%%%%%%%%%%%%
\documentclass[12pt,a4paper]{report}
\usepackage[utf8x]{inputenc}
\usepackage[IL2]{fontenc}
\usepackage{a4wide}
\usepackage[left=2.5cm, right=2.5cm, top=3cm, bottom=3cm]{geometry}

%%%%%%%%%%%%%%%%%%%%%%%%%%%%%%%%%%%
% Include required packages
%%%%%%%%%%%%%%%%%%%%%%%%%%%%%%%%%%%
\usepackage{amsmath,amssymb,amsthm}
\usepackage[usenames]{color}
\usepackage{nicefrac}
\usepackage{verbatim}
\usepackage{graphicx}
\usepackage{enumitem}
\usepackage{setspace}

\usepackage[pdftex,bookmarks=true,colorlinks,linkcolor=blue,urlcolor=blue,unicode]{hyperref}
\hypersetup{pdftitle=ODCleanStore}

%%%%%%%%%%%%%%%%%%%%%%%%%%%%%%%%%%%
% Additional formatting settings
%%%%%%%%%%%%%%%%%%%%%%%%%%%%%%%%%%%

%\pagestyle{plain}
\pagestyle{headings}
\linespread{1.1}

%%%%%%%%%%%%%%%%%%%%%%%%%%%%%%%%%%%
% Include macro definitions and package settings
%%%%%%%%%%%%%%%%%%%%%%%%%%%%%%%%%%%

%%%%%%%%%%%%%%%%%%%%%%%%%%%%%%%%%%%
% Indent settings for paragraphs, itemize and enumerate environment
%%%%%%%%%%%%%%%%%%%%%%%%%%%%%%%%%%%
\setitemize{noitemsep,topsep=2pt,leftmargin=30pt}
\setenumerate{noitemsep,topsep=2pt,leftmargin=30pt}
\setdescription{style=sameline}
%\setlength{\parindent}{0pt} % nastavuje odsazení prvniho radku
%\setlength{\parskip}{1.2ex plus 0.5ex minus 0.2ex} % odstup mezi odstavci
\pretolerance=5000

\newcommand{\moreindent}{\addtolength{\leftskip}{1.8em}}
\newcommand{\lessindent}{\addtolength{\leftskip}{1.8em}}
\newcommand{\suppressgaps}{\setlength{\parskip}{0pt}}

%%%%%%%%%%%%%%%%%%%%%%%%%%%%%%%%%%%
% Shortcuts for math mode
%%%%%%%%%%%%%%%%%%%%%%%%%%%%%%%%%%%
\newcommand{\coloneqq}{\mathrel{\mathop:}=}
\renewcommand{\O}{{\mathcal{O}}}
\newcommand{\etc}[2]{#1_1,\ldots,#1_#2}
\def\<#1>{\leavevmode\hbox{\it #1\/}} % usage: \<variable>

%%%%%%%%%%%%%%%%%%%%%%%%%%%%%%%%%%%
% Other shortcuts and style commands
%%%%%%%%%%%%%%%%%%%%%%%%%%%%%%%%%%%
\newcommand{\quot}[1]{``#1''}
\newcommand{\code}[1]{\texttt{#1}}
\newcommand{\varcode}[1]{\textit{\textless #1\textgreater}}
\newcommand{\vartext}[1]{\textit{#1}}
\newcommand{\tab}{\rule{30pt}{0pt}}
\newcommand{\term}[1]{\textit{#1}}
\newcommand{\todo}[1]{}
\newcommand{\importantterm}[1]{\textbf{#1}}

% These macros use a dirty trick to persuade LaTeX to typeset chapter headers
% more readably and not to leave plenty of space above them.
\makeatletter
\def\@makechapterhead#1{
  {\parindent \z@ \raggedright \normalfont
   \Huge\bfseries \thechapter. #1
   \par\nobreak
   \vskip 20\p@
}}
\def\@makeschapterhead#1{
  {\parindent \z@ \raggedright \normalfont
   \Huge\bfseries #1
   \par\nobreak
   \vskip 20\p@
}}
\makeatother

% Chapter that is not numbered but included in the contents
\def\chapwithtoc#1{
\chapter*{#1}
\addcontentsline{toc}{chapter}{#1}
} 

\newenvironment{enumeratei}
	{
		\begin{enumerate}
		\renewcommand{\labelenumi}{(\textit{\roman{enumi}})}
	}
	{
		\end{enumerate}
	}

\newenvironment{glossarylist}
	{\begin{description}[style=nextline,itemsep=8pt]}
	{\end{description}}

\newenvironment{configlist}
	{\begin{description}[style=nextline,font=\ttfamily]}
	{\end{description}}

\newenvironment{dirlist}
	{\begin{description}[style=sameline,font=\ttfamily]}
	{\end{description}}
	
%%%%%%%%%%%%%%%%%%%%%%%%%%%%%%%%%%%
% Package settings
%%%%%%%%%%%%%%%%%%%%%%%%%%%%%%%%%%%%

% lstlistings environment
% Defines a custom trivlisting environment
\lstset{basicstyle=\ttfamily\footnotesize,columns=flexible,
  frame=lines,float=ht,captionpos=b,aboveskip=1.5\bigskipamount}
\lstnewenvironment{trivlisting}
  {\lstset{basicstyle=\ttfamily,aboveskip=\medskipamount,frame=none}}
  {}

%%%%%%%%%%%%%%%%%%%%%%%%%%%%%%%%%%%
% Document-specific commands
%%%%%%%%%%%%%%%%%%%%%%%%%%%%%%%%%%%
\newcommand{\refusermanual}{User Manual\xspace}
\newcommand{\refadminmanual}{Administrator's \& Installation Manual\xspace}
\newcommand{\refprogrammersguide}{Programmer's Guide\xspace}
\newcommand{\configdefault}[1]{\newline\textit{Default value:}~\code{#1}}
\newcommand{\odcs}{ODCleanStore\xspace}
\newcommand{\QE}{Query Execution\xspace}
\newcommand{\CR}{Conflict Resolution\xspace}
\newcommand{\OI}{Linker\xspace}
\newcommand{\QA}{Quality Assessment\xspace}
\newcommand{\DN}{Data Normalization\xspace}
\newcommand{\FE}{Administration Frontend\xspace}
\newcommand{\dbodcs}[1]{\code{DB.ODCLEANSTORE.\uppercase{#1}}}
\newcommand{\reqparagraph}[1]{\paragraph{\textnormal{\textit{#1}}}}
\newcommand{\aggrq}{aggregate quality\xspace}

%%%%%%%%%%%%%%%%%%%%%%%%%%%%%%%%%%%
% Aligned enumeration tables
%%%%%%%%%%%%%%%%%%%%%%%%%%%%%%%%%%%
\newcommand{\enumtable}[1]
{
	\begin{table}[!ht]
		\begin{tabularx}{\linewidth}{>{\textbf\bgroup}l<{\egroup}X}
			#1
		\end{tabularx}
	\end{table}
}

%%%%%%%%%%%%%%%%%%%%%%%%%%%%%%%%%%%
% Aligned field description tables
%%%%%%%%%%%%%%%%%%%%%%%%%%%%%%%%%%%
\newcommand{\fieldtable}[1]
{
	\begin{table}[!ht]
		\begin{tabularx}{\linewidth}{|>{\textbf\bgroup}l<{\egroup}|p{3cm}|X|}
			\hline
			\textnormal{Required} & Field & Description \\
			\hline \hline
			#1 \\
			\hline
		\end{tabularx}
	\end{table}
}


%%%%%%%%%%%%%%%%%%%%%%%%%%%%%%%%%%%
% Optionally disable use of images.
% Convenient for export to .dvi when the documents contain images not supported in plain latex
% Uncomment to disable images.
%%%%%%%%%%%%%%%%%%%%%%%%%%%%%%%%%%%
%\renewcommand{\includegraphics}[2][1]{}

\newcommand{\version}{0.1}
\newcommand{\documentname}{User Manual}


\begin{document}

\title{ODCleanStore -- \documentname}

\begin{titlepage}
\begin{center}

\large
Charles University in Prague

\smallskip

Faculty of Mathematics and Physics

%\vspace{\stretch{1}}
%{\bf\Large SOFTWARE PROJECT}

\vspace{\stretch{3}}

\resizebox{0.5\linewidth}{!}{\bf\Huge ODCleanStore}

%\bigskip
%\resizebox{0.3\linewidth}{!}{Open Data store}

\vspace{\stretch{3}}
\begin{spacing}{1.5} 
{\bf\Huge \documentname}
\end{spacing}

\vspace{\stretch{1}}
Version \version\\
\today

\vspace{\stretch{15}}

\begin{tabular}{rl}

\textbf{Authors:} & Jan Michelfeit \\
& Du\v san Rychnovsk\'y\\
& Jakub Daniel\\
& Petr Jerman\\
& Tom\' a\v s Soukup\\
\noalign{\vspace{3mm}}
\textbf{Supervisor:} & RNDr. Tom\' a\v s Knap
\end{tabular}

\end{center}
\end{titlepage}

\newpage

\renewcommand{\contentsname}{Contents}
\tableofcontents
\bigskip

\newpage

%%%%%%%%%%%%%%%%%%%%%%%%%%%%%%%%%%%%%%%%%%%%%%%%%%%%%%%%%%%%%%%%%%%%%%%%%%%%%%

\chapwithtoc{Introduction}

\section{What is ODCleanStore}

Purpose.

\section{Linked Data Framework}

Context.

\section{Overview of features}

\section{Examples of usage}

Use cases, mention expected deployment (public contracts, students).

%%%%%%%%%%%%%%%%%%%%%%%%%%%%%%%%%%%%%%%%%%%%%%%%%%%%%%%%%%%%%%%%%%%%%%%%%%%%%%

\chapter{User Roles}

%%%%%%%%%%%%%%%%%%%%%%%%%%%%%%%%%%%%%%%%%%%%%%%%%%%%%%%%%%%%%%%%%%%%%%%%%%%%%%

\chapter{Administration Frontend}

\section{Administration Frontend Overview}

\section{User Accounts Administration}

\section{Transformer management}

\section{Pipeline management}

\section{Transformer policies}

\subsection*{Quality Assessment}

\subsection*{Data Normalization}

\subsection*{Object Identification}

\section{Ontology management}

\section{Other Settings}

\subsection*{Query Execution \& Conflict Resolution}

\subsection*{Prefix management}

%%%%%%%%%%%%%%%%%%%%%%%%%%%%%%%%%%%%%%%%%%%%%%%%%%%%%%%%%%%%%%%%%%%%%%%%%%%%%%

\chapter{Web Services}

\section{Web Services Overview}

\section{Data Publisher}

\section{Data Consumer}

A consumer of data stored in ODCleanStore can query the database through the \term{Output Webservice}. The Output Webservice can be queried for data about a given URI resource, queried by keywords or queried for metadata of a given named graph. Conflicts in data returned in response to a query are resolved and the data are fused using policies provided by the user or by the administrator.

Additionally, the user can access the data in the clean database directly using the SPARQL endpoint powered by Virtuoso. This way the data consumer can use the full power of the SPARQL query language, however conflict resolution and provenance tracking is not supported for this type of queries.
\todo{define clean database (in glossary?)}
\todo{specify where the SPARQL endpoint can be accessed}

\section*{Output Webservice}

The Output Webservice is a REST webservice which can be accessed using both GET and POST HTTP methods equivalently.

\subsection{Types of queries}

The Output Webservice can be queried for:

\begin{enumerate}
	\item a resource URI
	\item keyword(s)
	\item named graph metadata
\end{enumerate}

Table \ref{tbl:queryTypes} lists where each type of query can be accessed.

\begin{table}[h]
\centering
\caption{Types of queries}
\label{tbl:queryTypes}
\begin{tabular}{|l|l|p{8cm}|}
	\hline
	Query & URI & Example of a query \\
	\hline \hline
	URI & \varcode{host}\code{/uri} & \mbox{http://localhost:8087/uri} \mbox{?uri=http\%3A\%2F\%2Fexample.com} \\
	\hline
	Keyword & \varcode{host}\code{/keyword} & http://localhost:8087/keyword=keyword\\
	\hline
	Named graph metadata & \varcode{host}\code{/namedGraph} & \mbox{http://localhost:8087/namedGraph} \mbox{?uri=http\%3A\%2F\%2Fexample.com} \\
	\hline
\end{tabular}
\end{table} 

More information is available in the Query Execition specification. \todo{reference}


\subsection{Request format}

\chapwithtoc{References}

%%%%%%%%%%%%%%%%%%%%%%%%%%%%%%%%%%%%%%%%%%%%%%%%%%%%%%%%%%%%%%%%%%%%%%%%%%%%%%

\appendix

\chapter{Glossary}

\end{document}
