%%%%%%%%%%%%%%%%%%%%%%%%%%%%%%%%%%%
% Basic formatting and settings
%%%%%%%%%%%%%%%%%%%%%%%%%%%%%%%%%%%
\documentclass[12pt,a4paper]{report}
\usepackage[utf8x]{inputenc}
\usepackage[IL2]{fontenc}
\usepackage{a4wide}
\usepackage[left=2cm, right=2cm, top=2.5cm, bottom=2.5cm]{geometry}

%%%%%%%%%%%%%%%%%%%%%%%%%%%%%%%%%%%
% Include required packages
%%%%%%%%%%%%%%%%%%%%%%%%%%%%%%%%%%%
\usepackage{amsmath,amssymb,amsthm}
\usepackage[usenames]{color}
\usepackage{nicefrac}
\usepackage{verbatim}
\usepackage{graphicx}
\usepackage{enumitem}
\usepackage{setspace}
\usepackage{tabularx}
\usepackage{listings}
\usepackage[section]{placeins}

\usepackage[pdftex,bookmarks=true,colorlinks,linkcolor=blue,urlcolor=blue,unicode]{hyperref}
\hypersetup{pdftitle=ODCleanStore}

%%%%%%%%%%%%%%%%%%%%%%%%%%%%%%%%%%%
% Additional formatting settings
%%%%%%%%%%%%%%%%%%%%%%%%%%%%%%%%%%%

%\pagestyle{plain}
\pagestyle{headings}
\linespread{1.1}
\setcounter{secnumdepth}{3}
\setcounter{tocdepth}{2}

%%%%%%%%%%%%%%%%%%%%%%%%%%%%%%%%%%%
% Include macro definitions and package settings
%%%%%%%%%%%%%%%%%%%%%%%%%%%%%%%%%%%

%%%%%%%%%%%%%%%%%%%%%%%%%%%%%%%%%%%
% Indent settings for paragraphs, itemize and enumerate environment
%%%%%%%%%%%%%%%%%%%%%%%%%%%%%%%%%%%
\setitemize{noitemsep,topsep=2pt,leftmargin=30pt}
\setenumerate{noitemsep,topsep=2pt,leftmargin=30pt}
\setdescription{style=sameline}
%\setlength{\parindent}{0pt} % nastavuje odsazení prvniho radku
%\setlength{\parskip}{1.2ex plus 0.5ex minus 0.2ex} % odstup mezi odstavci

\newcommand{\moreindent}{\addtolength{\leftskip}{1.8em}}
\newcommand{\lessindent}{\addtolength{\leftskip}{1.8em}}
\newcommand{\suppressgaps}{\setlength{\parskip}{0pt}}

%%%%%%%%%%%%%%%%%%%%%%%%%%%%%%%%%%%
% Shortcuts for math mode
%%%%%%%%%%%%%%%%%%%%%%%%%%%%%%%%%%%
\newcommand{\coloneqq}{\mathrel{\mathop:}=}
\renewcommand{\O}{{\mathcal{O}}}
\newcommand{\etc}[2]{#1_1,\ldots,#1_#2}
\def\<#1>{\leavevmode\hbox{\it #1\/}} % usage: \<variable>

%%%%%%%%%%%%%%%%%%%%%%%%%%%%%%%%%%%
% Other shortcuts and style commands
%%%%%%%%%%%%%%%%%%%%%%%%%%%%%%%%%%%
\newcommand{\quot}[1]{``#1''}
\newcommand{\code}[1]{\texttt{#1}}
\newcommand{\varcode}[1]{\textit{\textless #1\textgreater}}
\newcommand{\vartext}[1]{\textit{#1}}
\newcommand{\tab}{\rule{30pt}{0pt}}
\newcommand{\term}[1]{\textit{#1}}
\newcommand{\todo}[1]{}
\newcommand{\importantterm}[1]{\textbf{#1}}

% These macros use a dirty trick to persuade LaTeX to typeset chapter headers
% more readably and not to leave plenty of space above them.
\makeatletter
\def\@makechapterhead#1{
  {\parindent \z@ \raggedright \normalfont
   \Huge\bfseries \thechapter. #1
   \par\nobreak
   \vskip 20\p@
}}
\def\@makeschapterhead#1{
  {\parindent \z@ \raggedright \normalfont
   \Huge\bfseries #1
   \par\nobreak
   \vskip 20\p@
}}
\makeatother

% Chapter that is not numbered but included in the contents
\def\chapwithtoc#1{
\chapter*{#1}
\addcontentsline{toc}{chapter}{#1}
} 

\newenvironment{enumeratei}
	{
		\begin{enumerate}
		\renewcommand{\labelenumi}{(\textit{\roman{enumi}})}
	}
	{
		\end{enumerate}
	}

\newenvironment{glossarylist}
	{\begin{description}[style=nextline,itemsep=8pt]}
	{\end{description}}

\newenvironment{configlist}
	{\begin{description}[style=nextline,font=\ttfamily]}
	{\end{description}}

\newenvironment{dirlist}
	{\begin{description}[style=sameline,font=\ttfamily]}
	{\end{description}}
	
%%%%%%%%%%%%%%%%%%%%%%%%%%%%%%%%%%%
% Package settings
%%%%%%%%%%%%%%%%%%%%%%%%%%%%%%%%%%%%

% lstlistings environment
% Defines a custom trivlisting environment
\lstset{basicstyle=\ttfamily\footnotesize,columns=flexible,
  frame=lines,float=ht,captionpos=b,aboveskip=1.5\bigskipamount}
\lstnewenvironment{trivlisting}
  {\lstset{basicstyle=\ttfamily,aboveskip=\medskipamount,frame=none}}
  {}

%%%%%%%%%%%%%%%%%%%%%%%%%%%%%%%%%%%
% Document-specific commands
%%%%%%%%%%%%%%%%%%%%%%%%%%%%%%%%%%%
\newcommand{\refusermanual}{User Manual\xspace}
\newcommand{\refadminmanual}{Administrator's \& Installation Manual\xspace}
\newcommand{\refprogrammersguide}{Programmer's Guide\xspace}
\newcommand{\configdefault}[1]{\newline\textit{Default value:}~\code{#1}}
\newcommand{\odcs}{ODCleanStore\xspace}
\newcommand{\reqparagraph}[1]{\paragraph{\textnormal{\textit{#1}}}}

%%%%%%%%%%%%%%%%%%%%%%%%%%%%%%%%%%%
% Aligned enumeration tables
%%%%%%%%%%%%%%%%%%%%%%%%%%%%%%%%%%%
\newcommand{\enumtable}[1]
{
	\begin{table}[!ht]
		\begin{tabularx}{\linewidth}{>{\textbf\bgroup}l<{\egroup}X}
			#1
		\end{tabularx}
	\end{table}
}

%%%%%%%%%%%%%%%%%%%%%%%%%%%%%%%%%%%
% Aligned field description tables
%%%%%%%%%%%%%%%%%%%%%%%%%%%%%%%%%%%
\newcommand{\fieldtable}[1]
{
	\begin{table}[!ht]
		\begin{tabularx}{\linewidth}{|>{\textbf\bgroup}l<{\egroup}|p{3cm}|X|}
			\hline
			\textnormal{Required} & Field & Description \\
			\hline \hline
			#1 \\
			\hline
		\end{tabularx}
	\end{table}
}


%%%%%%%%%%%%%%%%%%%%%%%%%%%%%%%%%%%
% Optionally disable use of images.
% Convenient for export to .dvi when the documents contain images not supported in plain latex
% Uncomment to disable images.
%%%%%%%%%%%%%%%%%%%%%%%%%%%%%%%%%%%
%\renewcommand{\includegraphics}[2][1]{}

\newcommand{\version}{0.1}
\newcommand{\documentname}{User Manual}

\begin{document}

\title{ODCleanStore -- \documentname}

\begin{titlepage}
\begin{center}

\large
Charles University in Prague

\smallskip

Faculty of Mathematics and Physics

%\vspace{\stretch{1}}
%{\bf\Large SOFTWARE PROJECT}

\vspace{\stretch{3}}

\resizebox{0.5\linewidth}{!}{\bf\Huge ODCleanStore}

%\bigskip
%\resizebox{0.3\linewidth}{!}{Open Data store}

\vspace{\stretch{3}}
\begin{spacing}{1.5} 
{\bf\Huge \documentname}
\end{spacing}

\vspace{\stretch{1}}
Release \version\\
\today

\vspace{\stretch{15}}

\begin{tabular}{rl}

\textbf{Authors:} & Jan Michelfeit \\
& Du\v san Rychnovsk\'y\\
& Jakub Daniel\\
& Petr Jerman\\
& Tom\' a\v s Soukup\\
\noalign{\vspace{3mm}}
\textbf{Supervisor:} & RNDr. Tom\' a\v s Knap
\end{tabular}

\end{center}
\end{titlepage}

\newpage

\renewcommand{\contentsname}{Contents}
\tableofcontents
\bigskip

\newpage

%%%%%%%%%%%%%%%%%%%%%%%%%%%%%%%%%%%%%%%%%%%%%%%%%%%%%%%%%%%%%%%%%%%%%%%%%%%%%%

\chapwithtoc{Introduction}

\section{What is ODCleanStore}

Purpose.

\section{Linked Data Framework}

Context.

\section{Overview of features}

\section{Examples of usage}

Use cases, mention expected deployment (public contracts, students).

%%%%%%%%%%%%%%%%%%%%%%%%%%%%%%%%%%%%%%%%%%%%%%%%%%%%%%%%%%%%%%%%%%%%%%%%%%%%%%

\chapter{User Roles}

Data consumers accessing the Output Webservice (see Section \ref{sec:outputWS}) do not need to have an account in ODCleanStore; these users have a special role User (USR). Other users working with ODCleanStore need to have an account and their permissions are based on the roles they are assigned.

\section[Administrator]{Administrator (ADM)}

	Administrator has privileges to manages user accounts, assign roles and manage system-wide settings such as
	\begin{itemize}
		\item transformers that can be used in pipelines created by pipeline creators,
		\item settings of the Output Webservice (default aggregation policies, etc.),
		\item URI prefixes that can be used in settings and queries.
	\end{itemize}

	In addition, the administrator is authorized to edit pipelines and rules created by pipeline creators.

	More information, e.g. about adding transformers, can be found in the related document Administrator's \& Installation Manual.

\section[Ontology Creator]{Ontology Creator (ONC)}
	The ontology creator can import and edit ontologies registered in the system. The ontology creator is also responsible for inserting mappings (\code{owl:sameAs} links) between ontologies.

\section[Pipeline Creator]{Pipeline Creator (PIC)}
	The pipeline creator can create input data processing pipelines. This includes creating new pipelines and assigning transformers to them (Section \ref{sec:pipelineManagement}) and also creating rules for the transformers (Section \ref{sec:transformerRules}).

	Every pipeline creator is allowed to create custom pipelines and rule groups for predefined transformers. The pipeline creator has a read-only access to other creators' pipelines and rules (and can use such rules in custom pipelines), however rules and pipelines can only be edited by their author. The only exception is the administrator, who can edit arbitrary pipelines and rule groups.
	
\section[Data Producer]{Data Producer (SCR)} \todo{SCR? DAP?}
  The data producer can use the Input Webservice (Section \ref{sec:inputWS}) to insert new data to ODCleanStore. The system keeps track of which data were inserted by which data producer.

\section[Data Consumer]{Data Consumer (USR)} \todo{USR? DAC?}
  The data consumer can use the Output Webservice (Section \ref{sec:outputWS}) to ask queries over the data in the clean database. This role is special in that users in this role do not need to have an account (any user using the Output Webservice is automatically assigned the USR role).

%%%%%%%%%%%%%%%%%%%%%%%%%%%%%%%%%%%%%%%%%%%%%%%%%%%%%%%%%%%%%%%%%%%%%%%%%%%%%%

\chapter{Administration Frontend}

\section{Administration Frontend Overview}

\section{User Accounts Administration}

\section{Transformer management}
\label{sec:transformerManagement}

\section{Pipeline management}
\label{sec:pipelineManagement}

\section{Transformer rules}
\label{sec:transformerRules}

\subsection*{Quality Assessment}

\subsection*{Data Normalization}

\subsection*{Object Identification}

\section{Ontology management}
\label{sec:ontologyManagement}

\section{Other Settings}

\subsection{Query Execution \& Conflict Resolution}
\label{sec:QEnCR}

\subsection{Prefix management}
\label{sec:frontendPrefixMgmt}

%%%%%%%%%%%%%%%%%%%%%%%%%%%%%%%%%%%%%%%%%%%%%%%%%%%%%%%%%%%%%%%%%%%%%%%%%%%%%%

\chapter{Web Services}

\section{Web Services Overview}

\section{Data Publisher}
\label{sec:inputWS}

\section{Data Consumer}
\label{sec:outputWS}

A consumer of data stored in ODCleanStore can query the database through the \term{Output Webservice}. The Output Webservice can be queried for data about a given URI resource, queried by keywords or queried for metadata of a given named graph. Conflicts in data returned in response to a query are resolved and the data are fused using policies provided by the user or by the administrator.

Additionally, the user can access the data in the clean database directly using the SPARQL endpoint powered by Virtuoso. This way the data consumer can use the full power of the SPARQL query language, however conflict resolution and provenance tracking is not supported for this type of queries.
\todo{define clean database (in glossary?)}
\todo{specify where the SPARQL endpoint can be accessed}

\section*{Output Webservice}

The Output Webservice is a REST webservice which can be accessed using both GET and POST HTTP methods equivalently.

\subsection{Types of queries}

The Output Webservice can be queried for:

\begin{enumerate}
	\item a resource URI
	\item keyword(s)
	\item named graph metadata
\end{enumerate}

Table \ref{tbl:queryTypes} lists where each type of query can be accessed.

\begin{table}[h]
\centering
\label{tbl:queryTypes}
\begin{tabular}{|l|l|p{7.5cm}|}
	\hline
	Query & URI & Example of a query \\
	\hline \hline
	URI & \varcode{host}\code{/uri} & \mbox{http://localhost:8087/uri} \mbox{?uri=http\%3A\%2F\%2Fexample.com} \\
	\hline
	Keyword & \varcode{host}\code{/keyword} & http://localhost:8087/keyword=keyword\\
	\hline
	Named graph metadata & \varcode{host}\code{/namedGraph} & \mbox{http://localhost:8087/namedGraph} \mbox{?uri=http\%3A\%2F\%2Fexample.com} \\
	\hline
\end{tabular}
\caption{Types of queries}
\end{table} 

More information is available in the Query Execition specification. \todo{reference}

\subsection{Request format}

The following table lists (either GET or POST) parameters than can be used with the URI and keyword queries:

\begin{table}[h!]
\centering
\label{tbl:requestFormatUK}
\begin{tabularx}{\textwidth}{|l|X|p{2cm}|p{2cm}|}
	\hline
	Name & Description & Possible values & Default value \\
	\hline \hline
	\code{uri} & searched URI; \newline \textit{used only with URI query} & \vartext{string} & \vartext{N/A} \\
	\hline
	\code{kw} & searched keyword(s); \newline \textit{used only with keyword query} & \vartext{string} & \vartext{N/A} \\
	\hline
	\code{format} & format of the result & \code{html}, \code{trig} & \code{html} \\
	\hline
	\code{aggr} & default aggregation method & \vartext{string} & \code{ALL} \\
	\hline
	\code{es} & error strategy -- handling of values for which aggregation fails & \code{IGNORE}, \code{RETURN\_ALL} & \code{RETURN\_ALL} \\
	\hline
	\code{multivalue} & default multivalue setting & 0, 1 & 0 \\
	\hline
	\code{paggr[\vartext{property}]} & aggregation method for the given property; \newline example: \code{paggr[rdfs\%3Alabel]=ANY} & \vartext{string} & \vartext{N/A} \\
	\hline
	\code{pmultivalue[\vartext{property}]} & multivalue setting for the given property;
		\newline example: \code{pmultivalue[rdf\%3Atype]=1} & 0, 1 & \vartext{N/A} \\
	\hline
\end{tabularx}
\caption{URI and keyword query parameters}
\end{table} 

The next table lists parameters that can be used with the named graph metadata query:

\begin{table}[h!]
\centering
\label{tbl:requestFormatNG}
\begin{tabular}{|l|l|l|l|}
	\hline
	Name & Description & Possible values & Default value \\
	\hline \hline
	\code{uri} & URI of the requested named graph & \vartext{string} & \vartext{N/A} \\
	\hline
	\code{format} & format of the result & \code{html}, \code{trig} & \code{html} \\
	\hline
\end{tabular}
\caption{Named graph metadata query parameters}
\end{table}

For all queries, parameters and values are case-sensitive. Property names may be either full URIs, or prefixed names (e.g. \code{rdfs:label}). Available prefixes are managed in the administration frontend (see section \ref{sec:frontendPrefixMgmt}).

For more information about aggregation settings, see the corresponding section of Conflict Resolution specification\todo{reference}.

\subsubsection*{General aggregation methods}

\begin{description}
	\item[ALL]
		returns all conflicting values
	\item[BEST]
		value with the highest aggregated quality; in case of equality, the newest timestamp is preferred
	\item[LATEST]
		value with the newest timestamp; in case of equality, the highest aggregate quality is preferred
	\item[ANY]
		returns a single arbitrary value
	\item[CONCAT]
		concatenation of conflicting values separated by \quot{\code{;\ }}
	\item[NONE]
		retruns all conflicting values including duplicities
\end{description}

\subsubsection*{Numeric aggregation methods}

\begin{description}
	\item[MIN]
		minimum of conflicting values
	\item[MAX]
		maximum of conflicting values
	\item[AVG]
		average of conflicting values
	\item[MEDIAN]
		median of conflicting values
\end{description}

\subsubsection*{Date aggregation methods}

\begin{description}
	\item[MIN]
		the earliest date
	\item[MAX]
		the latest date
\end{description}

\subsubsection*{String aggregation methods}

\begin{description}
	\item[SHORTEST]
		the shortest string
	\item[LONGEST]
		the longest string
\end{description}

\subsubsection*{Error strategy}
The error strategy determines how to handle values that cannot be aggregated by the given aggregation method, e.g. when applying MEDIAN aggregation to a mix of numeric and date values.

Note that for some aggregations, an untyped literal may be converted to a numeric literal (\code{xsd:double}) if possible.

\subsubsection*{Multivalie parameter}
The multivalue parameter determines whether differences with other conflicting values decrease quality (\code{multivalue=0}), or not (\code{multivalue=1}). Setting multivalue to false (0) is appropriate for properties with a single value (e.g. \code{dbprop:population}), setting it to true (1) is appropriate for propertiees with multiple possible values (e.g. \code{rdf:type}).

\todo{content-negotiation}

\subsection{Query Format}

\subsubsection{URI Query \& Named Graph Metadata Query}
The value of the \code{uri} parameter must be either a full valid URI, or a prefixed name (e.g. \code{dbpedia:Berlin}).  Available prefixes are managed in the administration frontend (see section \ref{sec:frontendPrefixMgmt}).

\subsubsection{Keyword Query}
The \code{kw} parameter can contain one or more keywords separated by whitespace. If a keyword itself contains spaces, it may be enclosed in double quotes. Query Execution looks for literals that contain all of the keywords. Keywords can also contain the \code{*} wildcard, but they must begin with at least four non-wildcard characters if a wildcard is to be used.

Query Execution also looks for an exact match of the entire \code{kw} value (i.e. without any division to keywords). If the \code{kw} value is a number, then numeric typed literals will match; if the \code{kw} value is formatted as \code{xsd:dateTime}\footnote{\url{http://www.w3.org/TR/xmlschema-2/\#dateTime-lexical-representation}}, then \code{xsd:dateTime} typed literals will match.

\subsection{Results Format for URI \& Keyword Queries}
The result contains triples returned in response to the query, including relevant labels of URI resources in the result, and metadata for the triples.

\subsubsection{HTML}

The result in HTML format contains results in a human-readable form. It contains

\begin{itemize}
	\item a table with all triples in the result together with their aggregated quality and named graphs from which the triple was selected or calculated,
  \item  a table with metadata of named graphs occuring in the first table.
\end{itemize}

\subsubsection{TriG}

The result contains triples (quads) serialized in the TriG\footnote{\url{http://www4.wiwiss.fu-berlin.de/bizer/trig/}} format. The result includes:

\begin{itemize}
	\item triples returned in response to the query, each one placed in a unique named graph
  \item aggregated quality (\code{odcs:quality}) and source named graphs (\code{w3p:source}) of the above triples; subjects of these statements are the unique named graphs where the respective triples are placed
  \item  metadata of source named graphs; they may include where the data were extracted from (\code{w3p:source}), Quality Assesment score of the named graph (\code{odcs:score}) and its publisher (\code{odcs:publisherScore}), the publisher of the data (\code{w3p:publishedBy}), timestamp (\code{w3p:insertedAt}), licence (\code{dc:licence})
  \item  metadata about the query response itself -- a title (\code{dc:title}), date (\code{dc:date}), number of result triples (\code{odcs:totalResults}), the query (\code{odcs:query}) and link to each result item (\code{odcs:result})
\end{itemize}

\pagebreak

An example:

\begin{lstlisting}[caption={Example of query response in TriG}]
@prefix :        <#> .
@prefix odcs:    <http://opendata.cz/infrastructure/odcleanstore/> .
@prefix w3p:     <http://purl.org/provenance#> .
@prefix rdfs:    <http://www.w3.org/2000/01/rdf-schema#> .
@prefix rdf:     <http://www.w3.org/1999/02/22-rdf-syntax-ns#> .
@prefix dc:      <http://purl.org/dc/terms/> .

<http://opendata.cz/infrastructure/odcleanstore/query/metadata/> {
  <http://opendata.cz/infrastructure/odcleanstore/data/e0cdc9d7-e2d8-4bde>
    w3p:insertedAt "2012-04-01 12:34:56.0"^^<http://www.w3.org/2001/XMLSchema#dateTime> ;
    w3p:source <http://dbpedia.org/page/Berlin> ;
    w3p:publishedBy <http://dbpedia.org/> ;
    odcs:provenanceMetadataGraph
      <http://opendata.cz/infrastructure/odcleanstore/provenanceMetadata/e0cdc9d7-e2d8-4bde>;
        
    odcs:score 0.72 ;
    odcs:violatedQARule <http://opendata.cz/infrastructure/odcleanstore/QARule/10> ;
    odcs:violatedQARule <http://opendata.cz/infrastructure/odcleanstore/QARule/20> .  

  <http://opendata.cz/infrastructure/odcleanstore/QARule/10>
    a odcs:QARule ;
    odcs:coefficient 0.8 ;
    dc:description "Procedure type ambiguous" .
        
  <http://opendata.cz/infrastructure/odcleanstore/QARule/20>
    a odcs:QARule ;
    odcs:coefficient 0.9 ;
    dc:description "Procurement contact person missing" .
        
  <http://localhost:8087/namedGraph?uri=http%3A%2F%2Fopendata.cz
      %2Finfrastructure%2Fodcleanstore%2Fdata%2Fe0cdc9d7-e2d8-4bde>
    a odcs:QueryResponse ;
    dc:title "Metadata for named graph:
      http://opendata.cz/infrastructure/odcleanstore/data/e0cdc9d7-e2d8-4bde" ;
    dc:date "2012-08-01T10:20:30+01:00" ;
    odcs:query "http://opendata.cz/infrastructure/odcleanstore/data/e0cdc9d7-e2d8-4bde";
}
    
<http://opendata.cz/infrastructure/odcleanstore/provenanceMetadata/e0cdc9d7-e2d8-4bde> {
  <http://opendata.cz/infrastructure/odcleanstore/data/e0cdc9d7-e2d8-4bde>
    w3p:provenanceMetadataProperty1 "provenanceMetadataValue1".
  <http://opendata.cz/infrastructure/odcleanstore/data/e0cdc9d7-e2d8-4bde>
    w3p:provenanceMetadataProperty2 "provenanceMetadataValue2".
}
\end{lstlisting}

\todo{JSON}
\todo{Error handling!}
\todo{Examples}


\chapwithtoc{References}

%%%%%%%%%%%%%%%%%%%%%%%%%%%%%%%%%%%%%%%%%%%%%%%%%%%%%%%%%%%%%%%%%%%%%%%%%%%%%%

\appendix

\chapter{Glossary}

\section*{RDF related}
\begin{glossarylist}
	\item[RDF] Resource Description Framework, a language for representing information about resources in the World Wide Web\footnote{\url{http://www.w3.org/RDF/}}
	\item[RDF triple] Statement about a resource expressed in the form of subject-predicate-object expression
	\item[URI] Uniform Resource Identifier, identifies RDF resources
	\item[Named graph] A set of related RDF triples (RDF graph) named with a URI
	\item[RDF quad] An RDF triple plus named graph URI (subject, predicate, object, named graph)
	\item[Ontology] Representation of the meaning of terms in a vocabulary and of their interrelationships
	\item[OWL] The Web Ontology Language\footnote{\url{http://www.w3.org/TR/owl-features/}}
\end{glossarylist}

\section*{Data \& Data Quality}
\begin{glossarylist}
	\item[Dirty (staging) database] Database where incoming data are stored until they are processed by a processing pipeline (e.g. clean, linked to other data, etc.)
	\item[Clean database] Database where incoming data are stored after they are successfully processed by the respective processing pipeline; this database can be accessed using the Output Webservice
	\item[Named graph score (\code{odcs:score})] Quality of a single named graph estimated by the Quality Assesment component and stored in the database, expressed as a number from interval [0,1]
	\item[Publisher score] Average score of named graphs from a publisher
	\item[Aggregate quality] Quality of a triple in the results calculated by the Conflict Resolution component during query time, expressed as a number from interval [0,1]
\end{glossarylist}

\section*{User Roles}
\begin{glossarylist}
	\item[ADM] Administrator
	\item[ONC] Ontology creator
	\item[PIC] Pipeline creator
	\item[SCR] Data producer (scraper)
	\item[USR] Data consumer
\end{glossarylist}

\chapter{List of Used XML Namespaces}

\begin{table}[h!]
\centering
\begin{tabular}{|l|l|}
	\hline
	\ttfamily
	\textrm{\textbf{Prefix}} & \textrm{\textbf{URI}} \\
	\hline\hline
	odcs & http://opendata.cz/infrastructure/odcleanstore/ \\
	\hline
	w3p & http://purl.org/provenance\# \\
	\hline
	dc & http://purl.org/dc/terms/ \\
	\hline
	rdf & http://www.w3.org/1999/02/22-rdf-syntax-ns\# \\
	\hline
	rdfs & http://www.w3.org/2000/01/rdf-schema\# \\
	\hline
	owl & http://www.w3.org/2002/07/owl\# \\
	\hline
	xsd & http://www.w3.org/2001/XMLSchema\# \\
	\hline
	dbpedia & http://dbpedia.org/resource/ \\
	\hline
	dbprop & http://dbpedia.org/property/ \\
	\hline
\end{tabular}
%\caption{List of used XML namespaces}
\end{table} 

\end{document}
