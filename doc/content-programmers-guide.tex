%%%%%%%%%%%%%%%%%%%%%%%%%%%%%%%%%%%
% Basic formatting and settings
%%%%%%%%%%%%%%%%%%%%%%%%%%%%%%%%%%%
\documentclass[12pt,a4paper]{report}
\usepackage[utf8x]{inputenc}
\usepackage[IL2]{fontenc}
\usepackage{a4wide}
\usepackage[left=2cm, right=2cm, top=2.5cm, bottom=2.5cm]{geometry}

%%%%%%%%%%%%%%%%%%%%%%%%%%%%%%%%%%%
% Include required packages
%%%%%%%%%%%%%%%%%%%%%%%%%%%%%%%%%%%
\usepackage{amsmath,amssymb,amsthm}
\usepackage[usenames]{color}
\usepackage{nicefrac}
\usepackage{verbatim}
\usepackage{graphicx}
\usepackage{enumitem}
\usepackage{setspace}
\usepackage{tabularx}
\usepackage{listings}
\usepackage[section]{placeins}

\usepackage[pdftex,bookmarks=true,colorlinks,linkcolor=blue,urlcolor=blue,unicode]{hyperref}
\hypersetup{pdftitle=ODCleanStore}

%%%%%%%%%%%%%%%%%%%%%%%%%%%%%%%%%%%
% Additional formatting settings
%%%%%%%%%%%%%%%%%%%%%%%%%%%%%%%%%%%

%\pagestyle{plain}
\pagestyle{headings}
\linespread{1.1}
\setcounter{secnumdepth}{3}
\setcounter{tocdepth}{2}

%%%%%%%%%%%%%%%%%%%%%%%%%%%%%%%%%%%
% Include macro definitions and package settings
%%%%%%%%%%%%%%%%%%%%%%%%%%%%%%%%%%%

%%%%%%%%%%%%%%%%%%%%%%%%%%%%%%%%%%%
% Indent settings for paragraphs, itemize and enumerate environment
%%%%%%%%%%%%%%%%%%%%%%%%%%%%%%%%%%%
\setitemize{noitemsep,topsep=2pt,leftmargin=30pt}
\setenumerate{noitemsep,topsep=2pt,leftmargin=30pt}
\setdescription{style=sameline}
%\setlength{\parindent}{0pt} % nastavuje odsazení prvniho radku
%\setlength{\parskip}{1.2ex plus 0.5ex minus 0.2ex} % odstup mezi odstavci

\newcommand{\moreindent}{\addtolength{\leftskip}{1.8em}}
\newcommand{\lessindent}{\addtolength{\leftskip}{1.8em}}
\newcommand{\suppressgaps}{\setlength{\parskip}{0pt}}

%%%%%%%%%%%%%%%%%%%%%%%%%%%%%%%%%%%
% Shortcuts for math mode
%%%%%%%%%%%%%%%%%%%%%%%%%%%%%%%%%%%
\newcommand{\coloneqq}{\mathrel{\mathop:}=}
\renewcommand{\O}{{\mathcal{O}}}
\newcommand{\etc}[2]{#1_1,\ldots,#1_#2}
\def\<#1>{\leavevmode\hbox{\it #1\/}} % usage: \<variable>

%%%%%%%%%%%%%%%%%%%%%%%%%%%%%%%%%%%
% Other shortcuts and style commands
%%%%%%%%%%%%%%%%%%%%%%%%%%%%%%%%%%%
\newcommand{\quot}[1]{``#1''}
\newcommand{\code}[1]{\texttt{#1}}
\newcommand{\varcode}[1]{\textit{\textless #1\textgreater}}
\newcommand{\vartext}[1]{\textit{#1}}
\newcommand{\tab}{\rule{30pt}{0pt}}
\newcommand{\term}[1]{\textit{#1}}
\newcommand{\todo}[1]{}
\newcommand{\importantterm}[1]{\textbf{#1}}

% These macros use a dirty trick to persuade LaTeX to typeset chapter headers
% more readably and not to leave plenty of space above them.
\makeatletter
\def\@makechapterhead#1{
  {\parindent \z@ \raggedright \normalfont
   \Huge\bfseries \thechapter. #1
   \par\nobreak
   \vskip 20\p@
}}
\def\@makeschapterhead#1{
  {\parindent \z@ \raggedright \normalfont
   \Huge\bfseries #1
   \par\nobreak
   \vskip 20\p@
}}
\makeatother

% Chapter that is not numbered but included in the contents
\def\chapwithtoc#1{
\chapter*{#1}
\addcontentsline{toc}{chapter}{#1}
} 

\newenvironment{enumeratei}
	{
		\begin{enumerate}
		\renewcommand{\labelenumi}{(\textit{\roman{enumi}})}
	}
	{
		\end{enumerate}
	}

\newenvironment{glossarylist}
	{\begin{description}[style=nextline,itemsep=8pt]}
	{\end{description}}

\newenvironment{configlist}
	{\begin{description}[style=nextline,font=\ttfamily]}
	{\end{description}}

\newenvironment{dirlist}
	{\begin{description}[style=sameline,font=\ttfamily]}
	{\end{description}}
	
%%%%%%%%%%%%%%%%%%%%%%%%%%%%%%%%%%%
% Package settings
%%%%%%%%%%%%%%%%%%%%%%%%%%%%%%%%%%%%

% lstlistings environment
% Defines a custom trivlisting environment
\lstset{basicstyle=\ttfamily\footnotesize,columns=flexible,
  frame=lines,float=ht,captionpos=b,aboveskip=1.5\bigskipamount}
\lstnewenvironment{trivlisting}
  {\lstset{basicstyle=\ttfamily,aboveskip=\medskipamount,frame=none}}
  {}

%%%%%%%%%%%%%%%%%%%%%%%%%%%%%%%%%%%
% Document-specific commands
%%%%%%%%%%%%%%%%%%%%%%%%%%%%%%%%%%%
\newcommand{\refusermanual}{User Manual\xspace}
\newcommand{\refadminmanual}{Administrator's \& Installation Manual\xspace}
\newcommand{\refprogrammersguide}{Programmer's Guide\xspace}
\newcommand{\configdefault}[1]{\newline\textit{Default value:}~\code{#1}}
\newcommand{\odcs}{ODCleanStore\xspace}
\newcommand{\reqparagraph}[1]{\paragraph{\textnormal{\textit{#1}}}}

%%%%%%%%%%%%%%%%%%%%%%%%%%%%%%%%%%%
% Aligned enumeration tables
%%%%%%%%%%%%%%%%%%%%%%%%%%%%%%%%%%%
\newcommand{\enumtable}[1]
{
	\begin{table}[!ht]
		\begin{tabularx}{\linewidth}{>{\textbf\bgroup}l<{\egroup}X}
			#1
		\end{tabularx}
	\end{table}
}

%%%%%%%%%%%%%%%%%%%%%%%%%%%%%%%%%%%
% Aligned field description tables
%%%%%%%%%%%%%%%%%%%%%%%%%%%%%%%%%%%
\newcommand{\fieldtable}[1]
{
	\begin{table}[!ht]
		\begin{tabularx}{\linewidth}{|>{\textbf\bgroup}l<{\egroup}|p{3cm}|X|}
			\hline
			\textnormal{Required} & Field & Description \\
			\hline \hline
			#1 \\
			\hline
		\end{tabularx}
	\end{table}
}


%%%%%%%%%%%%%%%%%%%%%%%%%%%%%%%%%%%
% Optionally disable use of images.
% Convenient for export to .dvi when the documents contain images not supported in plain latex
% Uncomment to disable images.
%%%%%%%%%%%%%%%%%%%%%%%%%%%%%%%%%%%
%\renewcommand{\includegraphics}[2][1]{}

\newcommand{\version}{0.3.1}
\newcommand{\documentname}{\refprogrammersguide}


\begin{document}

\title{ODCleanStore -- \documentname}

\begin{titlepage}
\begin{center}

\large
Charles University in Prague

\smallskip

Faculty of Mathematics and Physics

%\vspace{\stretch{1}}
%{\bf\Large SOFTWARE PROJECT}

\vspace{\stretch{3}}

\resizebox{0.5\linewidth}{!}{\bf\Huge ODCleanStore}

%\bigskip
%\resizebox{0.3\linewidth}{!}{Open Data store}

\vspace{\stretch{3}}
\begin{spacing}{1.5} 
{\bf\Huge \documentname}
\end{spacing}

\vspace{\stretch{1}}
Release \version\\
\today

\vspace{\stretch{15}}

\begin{tabular}{rl}

\textbf{Authors:} & Jan Michelfeit \\
& Du\v san Rychnovsk\'y\\
& Jakub Daniel\\
& Petr Jerman\\
& Tom\' a\v s Soukup\\
\noalign{\vspace{3mm}}
\textbf{Supervisor:} & RNDr. Tom\' a\v s Knap
\end{tabular}

\end{center}
\end{titlepage}

\newpage

\renewcommand{\contentsname}{Contents}
\tableofcontents
\bigskip

\newpage

%%%%%%%%%%%%%%%%%%%%%%%%%%%%%%%%%%%%%%%%%%%%%%%%%%%%%%%%%%%%%%%%%%%%%%%%%%%%%%

\chapwithtoc{Introduction}
Lorem ipsum.

\todo{že to je v javě}

\chapter{ODCleanStore overview} 

\begin{itemize}
	\item What is ODCleanStore
	\item architecture, architecture features, requirements
	\item Components overview
	\item Data processing overview
\end{itemize}

\section{Used technologies}

% https://sourceforge.net/p/odcleanstore/wiki/Technologies/

\begin{itemize}
	\item Java, platform (in)dependence
	\item RDF, Virtuoso, relational database
	\item references to other sources -- used formats (RDF/XML, TriG) etc.
	\item Maven
	\item libraries - Jena, \ldots
	\item Wicket
	\item webserver
\end{itemize}

\chapter{Setting up Development Environment}

\section{Quick Start}
\subsection{Tools}
	In order to to prepare environment for building ODCleanStore, first make sure to have installed all necessary tools:

\begin{itemize}
	\item Java Development Kit\footnote{\url{www.oracle.com/technetwork/java/javase/downloads/}} version 6\todo{} or newer
	\item Git version control system\footnote{\url{http://git-scm.com/}}
	\item Apache Maven\footnote{\url{http://maven.apache.org/}}
\end{itemize}

Also, make sure you have all the binaries used in the following steps on your classpath. 

\subsection{Obtaining sources} 
ODCleanStore sources are hosted at SourceForge.\footnote{\url{http://sourceforge.net/p/odcleanstore/code/}} Use git to check out the sources:

\begin{verbatim}
      git clone git://git.code.sf.net/p/odcleanstore/code odcleanstore-code
\end{verbatim}

The above command will create a local clone of the repository in \code{odcleanstore-code} directory.

\subsection{Building binaries}
Move to directory \code{odcleanstore} within your local clone of the repository which contains the root maven project (\code{pom.xml}). Then build the project using maven:

\begin{verbatim}
      cd odcleanstore-code/odcleanstore
      mvn clean package
\end{verbatim}

After that, Engine binaries can be found in directory \code{odcleanstore/engine/target} and the WAR file of Administration Frontend at \code{odcleanstore/webfrontend/target}. Now you can deploy the application as described in \refadminmanual. \todo{jestli tam bude jenom popis instalatoru, mel by se rucni deploy popsat tady a pridat referenci na instalaci Virtuosa}

\section{Repository structure}
\subsection{Branches}
There are several branches in the git repository. The latest development version of is on branch \code{master}. Then there are release branches for each release named \code{release-0.1.x}, \code{release-0.2.x} etc. which contain stable versions for  releases  packages. Each commit that where used to prepeare a release package is labeled with a tag \code{release}-\varcode{version}. Finally, there are feature branches prefixed with \code{feature-}.

Common development takes place on branch \code{master}. New feature branches are created for features that may or may not be accepted or that need time to be finished before they are applied to \code{master}. When whey are finished, they are merged to \code{master} and the branch may be removed in time. Release branches stem from \code{master} for every new major release. Fixes and modifications for minor releases take place on release branches and may be merged from/to \code{master}.

\subsection{Directory structure}
This is an outline of directory structure in the git repository:

\begin{dirlist}
	\item[data/] \mbox{}
		\begin{dirlist}
			\item[initial\_db\_import/] -- database import files
				\begin{dirlist}
					\item[clean\_db/] -- SQL files to be imported to the clean database
					\item[dirty\_db/] -- SQL files to be imported to the dirty database
				\end{dirlist}
			\item[odcs\_configuration/] -- the default ODCleanStore configuration file
			\item[virtuoso\_configuration/] -- configuration files for Virtuoso database instances
		\end{dirlist}
	\item[doc/] -- documentation sources (in \LaTeX)
	\item[odcleanstore/] -- Java sources
		\begin{dirlist}
			\item[backend/] -- sources of \code{odcs-backend} artifact (transformers, Query Execution, Conflict Resolution)
			\item[conf/] -- configuration files for development in Eclipse
			\item[core/] -- sources of \code{odcs-core} artifact (common code shared by other artifacts)
			\item[engine/]  -- sources of \code{odcs-engine} artifact (Engine component)
			\item[inputclient/]  -- sources of \code{odcs-imputclient} artifact (Java client for Input Webservice)
			\item[simplescraper/]  -- simple Input Webservice import tool
			\item[simpletransformer/]  -- example of a custom transformer
			\item[webfrontend/]  -- sources of Administration Frontend
			\item[pom.xml] -- the root maven POM file
		\end{dirlist}
\end{dirlist}


\section{Maven build}
%There are two profiles in addition to the default one in the root maven configuration file (pom.xml).

    %Profile javadoc - enables generation of javadoc (which is disabled in the default profile).
    %Profile systest - enables unit tests in systest/ subdirectories, which test functionality related to database. In order to run these tests, a new Virtuoso instance with settings as in /data/virtuoso_configuration/virtuoso.ini-test must be set up first (see Virtuoso installation guide )

%Maven build for the selected profile can be executed with command line option -P:

%mvn clean package -P javadoc
%mvn clean package -P systest

\chapter{Webservices \& Interfaces}

\chapter{Engine}

\chapter{Transformers}

\section{Custom Transformers}
\section{Quality Assessment}
\section{Data Normalization}
\section{Object Identification}
\section{Other Transformers}

\chapter{Query Execution}

\chapter{Conflict Resolution}

\chapter{Administration Frontend}

\chapter{(Conclusion, evaluation)}

\chapter{(Related Work)}

\chapter{Future Work}

\chapwithtoc{References}


%%%%%%%%%%%%%%%%%%%%%%%%%%%%%%%%%%%%%%%%%%%%%%%%%%%%%%%%%%%%%%%%%%%%%%%%%%%%%%

\appendix

\chapter{ODCS ontology}

\chapter{Relational database schema}

% TODO asi SQL create tables dotazy + popis významu jednotlivých tabulek v textu u každé komponenty, která je používá
% https://sourceforge.net/p/odcleanstore/wiki/Relational%20database%20documentation/

\chapter{(\quot{chronologicky popis prubehu praci na projektu})}

\chapter{(The Team)}


\end{document}
